\chapter{Entwickler Dokumentation}
\label{entw_docu}
Bla Bla

\section{Voraussetzungen}
\label{voraussertungen}
Für die Nutzung der App wird ein Backend Server benötigt. Um die Server Software auszuführen benötigen Sie:
\begin{itemize}
\item Node.js, hier zu finden \url{https://nodejs.org/en/} (die LTS Version ist zu bevorzugen)
\item Git
\end{itemize}
Für die Kompelierung der App benötigen Sie die folgenden Software-Tools:
\begin{itemize}
\item Flutter SDK, hier zu finden \url{https://flutter.dev/docs/get-started/install}
\item Git
\item JDK 8, hier zu finden \url{https://www.oracle.com/de/java/technologies/javase/javase-jdk8-downloads.html}
\item Android Studio, dies kann zusammen mit dem Android Studio heruntergeladen werden \url{https://developer.android.com/studio}
\item 
\end{itemize}
Laden Sie die Backend Server Software vom Github-Repository \url{https://github.com/amandajoelle/Backend} herunter. Sie können diese als Zip-Datei herunterladen oder mit dem Befehl \textit{git clone https://github.com/amandajoelle/Backend.git}.\\
Laden Sie sich anschließend die CIRS-APP herunter, vom Github-Repository \url{https://github.com/amandajoelle/Frontend}.

\section{Benutzung}
\label{benutzung}
Zuerst müssen die benötigten Abhängigkeiten heruntergeladen werden, hierfür muss der Befehl \textit{npm install} im Ordner Backend ausgeführt werden. Erstellen Sie im Backend-Ordner eine \textit{.env} Datei. Diese sollte mit Umgebungsvariablen initialisiert werden. Dies können Sie der README.md entnehmen.\\
Anschließend muss die Datenbank erstellt werden. Mithilfe des Befehles \textit{npm run database} wird eine Datenbank im Unterordner \textit{database} erstellt. Hier befindet sich nun die Datei \textit{cirs.db}, welche alle Datenbankeinträge beinhaltet. Abschließend muss die Backend Server Software nur noch mit dem Kommando \textit{npm run dev} gestartet werden.
\\[0.5cm]
Für die Kompilierung und Nutzung der CIRS-APP in einem Android Emulator, installieren Sie Android Studio und fügen Sie dem AVD einen Emulator hinzu.
\begin{figure}[hbt!]
\includegraphics[width=16cm]{AVD.png}
\caption{AVD Menü zum erstellen und starten von Android Emulatoren}
\label{fig:avd}
\end{figure}
Starten Sie den Android Emulator und für Sie in einem Terminal, im Frontend Ordner, den Befehl \textit{flutter run --release} aus.
\\[0.5cm]
Im Login Bildschirm der App, können Sie ich mit einem Beispiel Nutzer anmelden, um Fälle zu bearbeiten. Die Zugangsdaten sind:
\begin{itemize}
\item E-Mail: Mueller@cirs.de
\item Passwort: 123456789
\end{itemize}
Die App kann ebenfalls auf einem Android Smartphone ausgeführt werden. Hierfür muss in der Datei \textit{main.dart}, im Verzeichnis \textit{frontend/lib} die ServerUrl geändert werden.
\begin{lstlisting}[language=Java, caption=ServerUrl ersetzen, label=lst:url_ersetzen]
...
final String serverUrl = 'http://10.0.2.2:8080';

void main() {
  runApp(MaterialApp(home: MyApp()));
}
...
\end{lstlisting}
Ersetzen Sie die IP Adresse in der Konstanten \textit{serverUrl}, aus dem Listing \ref{lst:url_ersetzen}, für die Ihres Computers auf dem die Backend Server Software ausgeführt wird. Sowohl \textit{http://} als auch \textit{:8080} müssen Bestandteil der URL bleiben.
