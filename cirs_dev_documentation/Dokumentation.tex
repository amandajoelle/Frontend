\documentclass[a4paper, 12pt]{scrreprt}
\usepackage[left = 2.5cm,right = 2cm,bottom = 4cm]{geometry}
\usepackage[onehalfspacing]{setspace}
\usepackage[
	pdftitle={Dokumentation kürzeste Pfade},
	pdfsubject={},
	pdfauthor={Robin Meier, Stefan Manthey},
	pdfkeywords={},
	hidelinks
]{hyperref}

\usepackage[utf8]{inputenc}
\usepackage[ngerman]{babel}
\usepackage[T1]{fontenc}
\usepackage{graphicx,subfig}
\graphicspath{{img/}}
\usepackage{fancyhdr}
\usepackage{lmodern}
\usepackage{color}
\usepackage{amsfonts}
\usepackage{amsmath}
\usepackage{listings}
\usepackage{xcolor}
\usepackage[subfigure]{tocloft}
\usepackage{hyperref}
\fontfamily{vna}\selectfont

\definecolor{dkgreen}{rgb}{0,0.6,0}
\definecolor{gray}{rgb}{0.5,0.5,0.5}
\definecolor{mauve}{rgb}{0.58,0,0.82}

\lstdefinestyle{mystyle}{
	numbers=left,
	frame=single,
	basicstyle=\small,
	numbers=left,
	numberstyle=\tiny,
	numbersep=15pt,tabsize=4,
	flexiblecolumns=true,
	keywordstyle=\color{blue},
	commentstyle=\color{dkgreen}, 
	stringstyle=\color{mauve},
	numberstyle=\tiny\color{gray},
	language=Java,
	breaklines=true,
	breakatwhitespace=true,
	morekeywords={*,num,String,var,library,get,set}
}

\lstset{style=mystyle}

\usepackage[backend=bibtex, style=numeric]{biblatex}
%\addbibresource{Literatur.bib}

\pagestyle{fancy}
\lhead{}
\chead{}
\rhead{\slshape \leftmark}

\lfoot{}
\cfoot{\thepage}
\rfoot{}

\renewcommand{\headrulewidth}{0.4pt}
\renewcommand{\footrulewidth}{0pt}

\begin{document}

\pagestyle{empty}
\pagenumbering{Roman}

\begin{center}
\begin{tabular}{p{\textwidth}}

\begin{center}
\includegraphics[scale=1.5]{img/HTW_Berlin_Logo_farbig.jpg}
\end{center}

\\

\begin{center}
\LARGE{\textsc{
CIRS APP
}}
\end{center}

\\

\begin{center}
\large{
Dokumentation der mobilen Critical Incedent Reporting System App
}
\end{center}

\\

\begin{center}
\textbf{\large{Dokumentation}}
\end{center}

\begin{center}
vorgelegt von
\end{center}

\begin{center}
\large{\textbf{Amanda Joelle Dzukou Kom}} \\
\large{\textbf{Stefan Manthey}} \\
\large{\textbf{Michael Pientka}}
\end{center}

\begin{center}
geschrieben von
\end{center}

\begin{center}
Stefan Manthey
\end{center}

\begin{center}
\large{27.01.2021}
\end{center}

\\

\\

\end{tabular}
\end{center}

\cleardoubleoddpage
\pagestyle{fancy}

\renewcommand{\cftpartleader}{\cftdotfill{\cftdotsep}} % for parts
\renewcommand{\cftchapleader}{\cftdotfill{\cftdotsep}} % for chapters
\renewcommand{\cftsecleader}{\cftdotfill{\cftdotsep}} % for sections
\tableofcontents
\newpage
\listoffigures
\newpage
%\listoftables
%\newpage
\lstlistoflistings
\newpage

%Hier die Inhalte inkludieren
\pagenumbering{arabic}
\include{02_entwickler_dokumentation}


%Literaturverzeichnis
%\printbibliography[
%heading=bibintoc,
%title={Quellenverzeichnis}
%]

\end{document}
